\documentclass[acmsmall,screen,nonacm=true]{acmart}
\settopmatter{printacmref=false, printccs=false, printfolios=false}
\authorsaddresses{}

\usepackage{xspace}

\newcommand{\ol}[1]{\textcolor{blue}{\ifmmode \text{[OL: #1]}\else [OL: #1] \fi}}

\newcommand{\amaya}[0]{\textsc{Amaya}\xspace}


\begin{document}

\title{\amaya v2.0}
\subtitle{System description for SMT-COMP 2024}

%%
%% The "author" command and its associated commands are used to define
%% the authors and their affiliations.
%% Of note is the shared affiliation of the first two authors, and the
%% "authornote" and "authornotemark" commands
%% used to denote shared contribution to the research.
\author{Vojtěch Havlena}
\affiliation{
  \institution{Brno University of Technology}
  \city{Brno}
  \country{Czech Republic}
  }
% \email{trovato@corporation.com}
% \orcid{1234-5678-9012}
% \author{G.K.M. Tobin}
% \authornotemark[1]
% \email{webmaster@marysville-ohio.com}
% \affiliation{%
%   \institution{Institute for Clarity in Documentation}
%   \city{Dublin}
%   \state{Ohio}
%   \country{USA}
% }

\author{Michal Hečko}
\affiliation{
  \institution{Brno University of Technology}
  \city{Brno}
  \country{Czech Republic}
  }
% \affiliation{%
%   \institution{The Th{\o}rv{\"a}ld Group}
%   \city{Hekla}
%   \country{Iceland}}
% \email{larst@affiliation.org}

\author{Lukáš Holík}
\affiliation{
  \institution{Brno University of Technology}
  \city{Brno}
  \country{Czech Republic}
  }
% \affiliation{%
%   \institution{Inria Paris-Rocquencourt}
%   \city{Rocquencourt}
%   \country{France}
% }

\author{Ondřej Lengál}
% \affiliation{%
%  \institution{Rajiv Gandhi University}
%  \city{Doimukh}
%  \state{Arunachal Pradesh}
%  \country{India}}

\affiliation{
  \institution{Brno University of Technology}
  \city{Brno}
  \country{Czech Republic}
  }

% \renewcommand{\shortauthors}{Trovato et al.}

\begin{abstract}
  This is a~brief overview of \amaya's submission to SMT-COMP 2024.
  \amaya is a standalone solver for quantified linear integer arithmetic (LIA) based on finite automata.
  It implements the standard automata-based decision procedure with several crucial optimizations that prune vast fractions of the underlying state space.
  Contrary to standard approaches, \amaya constructs a~compact representation of all models of the LIA formula.
\end{abstract}

\maketitle


%%%%%%%%%%%%%%%%%%%%%%%%%%%%%%%%%%%%%%%%%%%%%%%%%%%%%%%%%%%%%%%%%%%%%%%%%%%%%%%%
\vspace{-0.0mm}
\section{Overview}
\vspace{-0.0mm}
%%%%%%%%%%%%%%%%%%%%%%%%%%%%%%%%%%%%%%%%%%%%%%%%%%%%%%%%%%%%%%%%%%%%%%%%%%%%%%%%

\amaya is a~standalone solver for quantified \emph{linear integer arithmetic}
(LIA)\footnote{also sometimes referred to as \emph{Presburger arithmetic}}
based on finite automata (FAs).
The foundation of the decision procedure implemented in \amaya 
is the work of B\"{u}chi~\cite{Buchi60} (in the context of showing
automata-based decidability of the \emph{weak monadic second-order theory of
one successor}---WS1S---into which LIA can be encoded), which was later
tailored for LIA (including the linear congrurence operator) in the
works~\cite{BoudetC96,BoigelotJW05,Durand-GasselinH10}.
% and implemented in \lash~\cite{LASH}.
The basic procedure works by constructing an FA~$\autof \varphi$ for each atomic formula~$\varphi$, which
is of the form $a_1 x_1 + \ldots a_n x_n \sim c$ where ${\sim} \in \{{=},
{\leq}, {\equiv_m}\}$ for integers $a_1, \ldots, a_n, c$ and a natural
number~$m$.
The FA~$\autof \varphi$ represents all solutions of the formula~$\varphi$,
encoded as tuples of binary numbers in the two's complement encoding.

\amaya implements the procedure with several crucial optimizations described in~\cite{HabermehlHHHL24}.


\ol{}

Sylvan~\cite{DijkP15}

CAV \cite{HabermehlHHHL24}


%%%%%%%%%%%%%%%%%%%%%%%%%%%%%%%%%%%%%%%%%%%%%%%%%%%%%%%%%%%%%%%%%%%%%%%%%%%%%%%%
\vspace{-0.0mm}
\section{Approach}\label{sec:label}
\vspace{-0.0mm}
%%%%%%%%%%%%%%%%%%%%%%%%%%%%%%%%%%%%%%%%%%%%%%%%%%%%%%%%%%%%%%%%%%%%%%%%%%%%%%%%






% Biblography
\bibliographystyle{ACM-Reference-Format}
\bibliography{literature}



\end{document}
\endinput
