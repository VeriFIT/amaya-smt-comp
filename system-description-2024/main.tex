\documentclass[acmsmall,screen,nonacm=true]{acmart}
\settopmatter{printacmref=false, printccs=false, printfolios=false}
\authorsaddresses{}

\usepackage{xspace}

\newcommand{\ol}[1]{\textcolor{blue}{\ifmmode \text{[OL: #1]}\else [OL: #1] \fi}}

\newcommand{\amaya}[0]{\textsc{Amaya}\xspace}


\begin{document}

\title{\amaya v2.0}
\subtitle{System description for SMT-COMP 2024}

%%
%% The "author" command and its associated commands are used to define
%% the authors and their affiliations.
%% Of note is the shared affiliation of the first two authors, and the
%% "authornote" and "authornotemark" commands
%% used to denote shared contribution to the research.
\author{Vojtěch Havlena}
\affiliation{
  \institution{Brno University of Technology}
  \city{Brno}
  \country{Czech Republic}
  }
% \email{trovato@corporation.com}
% \orcid{1234-5678-9012}
% \author{G.K.M. Tobin}
% \authornotemark[1]
% \email{webmaster@marysville-ohio.com}
% \affiliation{%
%   \institution{Institute for Clarity in Documentation}
%   \city{Dublin}
%   \state{Ohio}
%   \country{USA}
% }

\author{Michal Hečko}
\affiliation{
  \institution{Brno University of Technology}
  \city{Brno}
  \country{Czech Republic}
  }
% \affiliation{%
%   \institution{The Th{\o}rv{\"a}ld Group}
%   \city{Hekla}
%   \country{Iceland}}
% \email{larst@affiliation.org}

\author{Lukáš Holík}
\affiliation{
  \institution{Brno University of Technology}
  \city{Brno}
  \country{Czech Republic}
  }
% \affiliation{%
%   \institution{Inria Paris-Rocquencourt}
%   \city{Rocquencourt}
%   \country{France}
% }

\author{Ondřej Lengál}
% \affiliation{%
%  \institution{Rajiv Gandhi University}
%  \city{Doimukh}
%  \state{Arunachal Pradesh}
%  \country{India}}

\affiliation{
  \institution{Brno University of Technology}
  \city{Brno}
  \country{Czech Republic}
  }

% \renewcommand{\shortauthors}{Trovato et al.}

\begin{abstract}
  This is a~brief overview of \amaya's submission to SMT-COMP 2024.
  \amaya is a standalone solver for quantified linear integer arithmetic (LIA) based on finite automata.
  It implements the standard automata-based decision procedure with several crucial optimizations that prune vast fractions of the underlying state space.
  Contrary to standard approaches, \amaya constructs a~compact representation of all models of the LIA formula.
\end{abstract}

\maketitle


%%%%%%%%%%%%%%%%%%%%%%%%%%%%%%%%%%%%%%%%%%%%%%%%%%%%%%%%%%%%%%%%%%%%%%%%%%%%%%%%
\vspace{-0.0mm}
\section{Overview}
\vspace{-0.0mm}
%%%%%%%%%%%%%%%%%%%%%%%%%%%%%%%%%%%%%%%%%%%%%%%%%%%%%%%%%%%%%%%%%%%%%%%%%%%%%%%%

\amaya\footnote{\url{https://github.com/MichalHe/amaya}} is a~standalone solver for
quantified \emph{linear integer arithmetic}
(LIA)\footnote{also sometimes referred to as \emph{Presburger arithmetic}}
based on finite automata (FAs).
The foundation of the decision procedure implemented in \amaya 
is the work of B\"{u}chi~\cite{Buchi60} (in the context of showing
automata-based decidability of the \emph{weak monadic second-order theory of
one successor}---WS1S---into which LIA can be encoded), which was later
tailored for LIA (including the linear congrurence operator) in the
works~\cite{BoudetC96,BoigelotJW05,Durand-GasselinH10}.
% and implemented in \lash~\cite{LASH}.
The basic procedure works by constructing an FA~$\autof{\varphi_\atom}$ for
each atomic formula~$\varphi_\atom$, which
is of the form $a_1 x_1 + \ldots a_n x_n \sim c$ where ${\sim} \in \{{=},
{\leq}, {\equiv_m}\}$ (with $\equiv_m$ being the linear congruence modulo~$m \in \nat \setminus \{0\}$)
for integers $a_1, \ldots, a_n, c$ and a natural number~$m$.
The FA~$\autof{\varphi_\atom}$ represents all solutions of the formula~$\varphi_\atom$,
encoded as tuples of binary numbers in the two's complement encoding.
FAs for complex formulae are then obtained inductively by performing standard
automata operations: \emph{intersection} for conjunction, \emph{union} for disjunction,
\emph{complementation} for negation.
Existential quantification is performed by the operation of \emph{projection}
on the alphabet of the corresponding FA (followed by saturation of the FAs
accepting states).

\amaya implements the procedure with several crucial optimizations described
in~\cite{HabermehlHHHL24}.
In particular, it fights the issue of constructing FAs with too many states
(which emerge, e.g., when translating modulo congruence predicates) by directly
creating FAs for non-atomic formulae using a~derivative based construction with
implicit pruning and subsumption.
Moreover, the procedure also uses certain techniques of quantifier
instantiation (based on formula monotonicity, finite range of models, and
modulo linearization) as well as standard simplification techniques for
on-the-fly rewriting of the formula for which the FA is being constructed. 
Before the main procedure is executed, \amaya also runs preprocessing, which
optimizes the input formula by static analysis tailored for the automata-based
procedure.

%%%%%%%%%%%%%%%%%%%%%%%%%%%%%%%%%%%%%%%%%%%%%%%%%%%%%%%%%%%%%%%%%%%%%%%%%%%%%%%%
\vspace{-0.0mm}
\section{Implementation}\label{sec:label}
\vspace{-0.0mm}
%%%%%%%%%%%%%%%%%%%%%%%%%%%%%%%%%%%%%%%%%%%%%%%%%%%%%%%%%%%%%%%%%%%%%%%%%%%%%%%%

The frontend of \amaya is implemented in Python.
Since the size of the underlying FA alphabet grows exponentially with the
number of variables in the formula, needing to deal with alphabets having
thousands up to millions of symbols, causing an explicit representation of the
transition relation infeasible.
To deal with this issue, \amaya encodes the transition relation using
\emph{multi-terminal binary decision diagrams} (MTBDDs) implemented within the
\sylvan library~\cite{DijkP15}, which often yields a~much more compact representation.
As \sylvan is a~C~library, \amaya includes a~C++ backend that connects to the
Python frontend via a~\texttt{ctypes} interface. The C++ backend also implements
procedures for constructing FA~$\autof{\varphi_\atom}$ accepting the solutions
of an atomic formula $\varphi_\atom$, and the derivative-based
construction.
On the other hand, the Python frontend performs parsing of the input SMTLIB
formula, preprocessing of the input by static analysis and rewriting, and
orchestration of the decision procedure (which is performed in the backend).

% Note(mhecko):
% The following is implemented in Python: SMTLIB parser, orchestration of the classical procedure
% (scheduling of automata operations based on corresponding Boolean connectives), rewriting of
% nonliner terms (mod terms, div terms), formula optimizations based on static analysis of the formula.



%%%%%%%%%%%%%%%%%%%%%%%%%%%%%%%%%%%%%%%%%%%%%%%%%%%%%%%%%%%%%%%%%%%%%%%%%%%%%%%%
\vspace{-0.0mm}
\section{\amaya in SMT-COMP 2024}\label{sec:label}
\vspace{-0.0mm}
%%%%%%%%%%%%%%%%%%%%%%%%%%%%%%%%%%%%%%%%%%%%%%%%%%%%%%%%%%%%%%%%%%%%%%%%%%%%%%%%


We are submitting version v2.0 of \amaya to compete in the single query track
of the division Arith, in logics LIA and NIA (nonlinear integer arithmetic; we
participate in NIA because it contains many formulae that are linear in nature, but
include the~$\equiv_m$ predicate, which is not considered linear by SMT-LIB).






% Biblography
\bibliographystyle{ACM-Reference-Format}
\bibliography{literature}



\end{document}
\endinput
